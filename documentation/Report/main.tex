\documentclass[12pt]{article}
\usepackage[pdftex]{graphicx} % Required for inserting images
\usepackage[T1]{fontenc}
\usepackage[polish]{babel}
\usepackage{amsmath}
\usepackage{geometry}
\usepackage{float}
\usepackage{hyperref}
\usepackage{polski}
\usepackage[utf8]{inputenc}
\usepackage{gensymb}
\usepackage{longtable}
\usepackage{array} % required for text wrapping in tables
\usepackage{algpseudocode}
\usepackage{algorithm}
\usepackage{indentfirst}
\usepackage{verbatim}
\usepackage{amsthm}
\usepackage{amsfonts}
\usepackage{thmtools}
\theoremstyle{definition}
\newtheorem{definition}{Definicja}
\begin{document}

\title{
  \textbf{Teoria algorytmów i obliczeń} \\
  \large Projekt zaliczeniowy}

\author
{
    Piotr Jacak \\
    Jakub Kindracki \\
    Wiktor Kobielski \\
    Ernest Mołczan \\
    \\
    Koordynator: prof. dr hab. inż. Władysław Homenda
    \\
    \\
}

\date{Semestr zimowy 2025/2026}



\begin{figure}
    \centering
    \includegraphics[width=1\linewidth]{mini.png}
\end{figure}

\maketitle

\pagebreak
\tableofcontents
\pagebreak

% ---------------- WSTĘP ---------------- %
\section{Wstęp}
\label{sec:wstep}

Niniejsza praca stanowi sprawozdanie z projektu zrealizowanego w ramach przedmiotu \textbf{Teoria algorytmów i obliczeń}. Przedmiotem badań są algorytmy operujące na multigrafach, ze szczególnym uwzględnieniem problematyki izomorfizmu podgrafów oraz minimalnych rozszerzeń grafów.

Głównym celem projektu jest opracowanie, analiza teoretyczna oraz implementacja algorytmów rozwiązujących dwa ściśle powiązane problemy. Pierwszym z nich jest weryfikacja, czy dany multigraf $H$ jest izomorficzny z $n$ podgrafami multigrafu $G$. Drugim, kluczowym zagadnieniem, jest wyznaczenie \emph{minimalnego rozszerzenia} multigrafu $G$ do postaci $G'$, która zawiera co najmniej $n$ podgrafów izomorficznych z $H$.

Realizacja powyższych celów wymagała formalnego zdefiniowania oraz uzasadnienia kilku fundamentalnych pojęć. W pracy zaproponowano autorskie lub bazujące na literaturze definicje:
\begin{itemize}
    \item \emph{rozmiaru multigrafu},
    \item \emph{metryki} w zbiorze multigrafów,
    \item \emph{minimalnego rozszerzenia} multigrafu.
\end{itemize}
Pojęcia te stanowią podstawę do dalszej analizy algorytmicznej oraz oceny kosztu operacji.

W ramach pracy przeprowadzono analizę złożoności obliczeniowej opracowanych algorytmów. Zgodnie z założeniami projektu, w przypadku gdy algorytmy dokładne charakteryzują się złożonością wykładniczą, przedstawiono również propozycje algorytmów aproksymacyjnych o złożoności wielomianowej.

% Niniejszy raport, oprócz formalnych definicji i analizy algorytmów, zawiera także opis przeprowadzonych testów obliczeniowych, dokumentację techniczną implementacji oraz wnioski końcowe.

\pagebreak
% ---------------- WSTĘP ---------------- %

% ---------------- DEFINICJE ---------------- %
\section{Definicje pojęć}
\label{sec:definicje}

\begin{definition}[Graf]
    Grafem nazywamy parę $G = (V, E)$, gdzie $V$ jest zbiorem wierzchołków, a $E \subseteq V \times V = \{(u, v) : u, v \in V \land u \neq v \}$ jest zbiorem krawędzi. Dla każdej pary wierzchołków $u, v \in V$ istnieje co najwyżej jedna krawędź łącząca wierzchołki $u$ i $v$.
\end{definition}

\begin{definition}[Multigraf]
    Multigrafem nazywamy graf, w którym pomiędzy dowolnymi dwoma różnymi wierzchołkami $u, v \in V$ może istnieć więcej niż jedna krawędź.
\end{definition}

\begin{definition}[Graf skierowany]
    Grafem skierowanym nazywamy parę $G = (V, E)$, gdzie $V$ jest zbiorem wierzchołków, a $E \subseteq V \times V = \{(u, v) : u, v \in V \land u \neq v \}$ jest zbiorem krawędzi. Krawędzie w grafie skierowanym mają określony kierunek, co oznacza, że krawędź $(u, v)$ jest różna od krawędzi $(v, u)$. Definicja jest analogiczna dla multigrafów.
\end{definition}

\begin{definition}[Izomorfizm grafów]
    Dwa grafy $G_1 = (V_1, E_1)$ i $G_2 = (V_2, E_2)$ są izomorficzne, wtedy i tylko wtedy, gdy istnieje bijekcja $f: V_1 \to V_2$, taka że dla każdej krawędzi $(u, v) \in E_1$ zachodzi $(f(u), f(v)) \in E_2$. Definicja ta jest analogiczna dla multigrafów i grafów skierowanych.
\end{definition}

\begin{definition}[Podgraf]
    Graf $H = (V_H, E_H)$ nazywamy podgrafem grafu $G = (V_G, E_G)$, wtedy i tylko wtedy, gdy $V_H \subseteq V_G$ oraz $E_H \subseteq E_G$. Definicja ta jest analogiczna dla multigrafów i grafów skierowanych.
\end{definition}

\begin{definition}[Graf atrybutowy]
    Graf $G = (V, E, f)$ nazywamy grafem atrybutowym, gdzie $V$ jest zbiorem wierzchołków, a $E \subseteq V \times V = \{(u, v) : u, v \in V \land u \neq v \}$ jest zbiorem krawędzi. Dla każdej pary wierzchołków $u, v \in V$ istnieje co najwyżej jedna krawędź łącząca wierzchołki $u$ i $v$. $f: E \to \Sigma_E $ jest funkcją, przypisującą etykiety wszystkim krawędziom w grafie $G$. 
\end{definition}

\begin{definition}[Macierz sąsiedztwa]
    Macierzą sąsiedztwa multigrafu $G = (V, E)$ nazywamy macierz $A$, której pole $A_{uv} = k$, wtedy i tylko wtedy, gdy istnieje $k$ krawędzi $(u, v) \in E$. W przypadku gdy nie istnieje żadna krawędź pomiędzy wierzchołkami $u$ i $v$, to $A_{uv} = 0$.
\end{definition}

\pagebreak
% ---------------- DEFINICJE ---------------- %

% ---------------- ROZMIAR MULTIGRAFU ---------------- %
\section{Rozmiar multigrafu}
\label{sec:rozmiar}

\begin{definition}[Rozmiar multigrafu]
    Rozmiarem $S(G)$ multigrafu $G = (V, E)$ nazywamy parę liczb naturalnych $(|V|, |E|)$, gdzie $|V|$ oznacza liczbę wierzchołków, a $|E|$ liczbę krawędzi w multigrafie $G$.
\end{definition}

Zakładamy, że liczby wierzchołków i krawędzi są zapisanymi wcześniej stałymi, więc obliczenie rozmiaru multigrafów jest operacją o złożoności czasowej $O(1)$.

\begin{definition}[Porządek w zbiorze wszystkich multigrafów]
    Niech $G_1$ i $G_2$ będą dwoma multigrafami. Mówimy, że $G_1$ jest mniejszy, lub równy $G_2$ wtedy i tylko wtedy, gdy:
    \[ |V_1| < |V_2| \lor (|V_1| = |V_2| \land |E_1| \leq |E_2|) \]
\end{definition}

Żeby udowodnić poprawność powyższej definicji porządku wykazujemy, że spełnia ona trzy wymagane własności:
\begin{itemize}
    \item \textbf{Zwrotność}: 
    \[S(G) \leq S(G)\]
    Dla dowolnego multigrafu $G = (V, E)$, zachodzi $|V| = |V| \land |E| = |E|$. Więc w szczególności spełnia on warunek $|V| = |V| \land |E| \leq |E|$ z definicji porządku. Stąd $S(G) \leq S(G)$.
    \item \textbf{Przechodniość}:
    \[S(G_1) \leq S(G_2) \land S(G_2) \leq S(G_3) \Rightarrow S(G_1) \leq S(G_3)\]
    Weźmy dowolne trzy multigrafy $G_1 = (V_1, E_1)$, $G_2 = (V_2, E_2)$ oraz $G_3 = (V_3, E_3)$ takie, że $S(G_1) \leq S(G_2)$ oraz $S(G_2) \leq S(G_3)$. 
    
    Załóżmy, że $S(G_1) \geq S(G_3)$. Z definicji to implikuje, że $|V_1| > |V_3| \lor (|V_1| = |V_3| \land |E_1| > |E_3|)$. 
    
    Z założeń wiemy też, że $|V_2| > |V_1|$, lub $|V_2| = |V_1| \land |E_2| \geq |E_1|$. 
    
    W pierwszym przypadku z założeń wynika, ze $|V_2| > |V_3|$, co stoi w sprzeczności z $S(G_2) \leq S(G_3)$. 
    
    W drugim przypadku, z założeń wynika, że $|V_2| = |V_3|$ oraz $|E_2| > |E_3|$, co również stoi w sprzeczności z $S(G_2) \leq S(G_3)$. 
    
    W obu przypadkach dochodzimy do sprzeczności, więc nasze początkowe założenie było fałszywe. Stąd $S(G_1) \leq S(G_3)$.
    \item \textbf{Antysymetryczność}: 
    \[S(G_1) \leq S(G_2) \land S(G_2) \leq S(G_1) \Rightarrow S(G_1) = S(G_2)\]
    Weźmy dowolne dwa multigrafy $G_1 = (V_1, E_1)$ oraz $G_2 = (V_2, E_2)$ takie, że $S(G_1) \leq S(G_2)$ oraz $S(G_2) \leq S(G_1)$. Z definicji porządku, z pierwszego założenia wynika, że $|V_1| < |V_2| \lor (|V_1| = |V_2| \land |E_1| \leq |E_2|)$. Z drugiego założenia wynika, że $|V_2| < |V_1| \lor (|V_2| = |V_1| \land |E_2| \leq |E_1|)$.

    Jeśli $|V_1| < |V_2|$, to z drugiego założenia wynika, że $|V_2| < |V_1|$, co jest sprzeczne. Analogicznie, jeśli $|V_2| < |V_1|$, to z pierwszego założenia wynika, że $|V_1| < |V_2|$, co również jest sprzeczne. Zatem musi zachodzić $|V_1| = |V_2|$.
    
    Wtedy z pierwszego założenia wynika, że $|E_1| \leq |E_2|$, a z drugiego, że $|E_2| \leq |E_1|$. Stąd $|E_1| = |E_2|$.
    
    W rezultacie mamy $S(G_1) = S(G_2)$.

\end{itemize}

\pagebreak
% ---------------- ROZMIAR MULTIGRAFU ---------------- %

% ---------------- METRYKA W ZBIORZE WSZYSTKICH MULTIGRAFÓW ---------------- %
\section{Metryka w zbiorze wszystkich multigrafów}
\label{sec:metryka}

\begin{definition}[Metryka w zbiorze multigrafów]
    Niech $\mathcal{G}$ będzie zbiorem wszystkich multigrafów. \textbf{Metryką} w zbiorze $\mathcal{G}$ nazywamy funkcję:
    \[d: \mathcal{G} \times \mathcal{G} \to \mathbb{N}_0\]
    Wartość $d(G_1, G_2)$ nazywamy \textbf{odległością} między multigrafami $G_1$ i $G_2$, a definiujemy ją, jako \textbf{minimalną} liczbę operacji dodawania lub usuwania pojedynczej krawędzi lub wierzchołka, za pomocą których można przekształcić graf $G_1$ w graf izomorficzny z $G_2$.
\end{definition}

% TODO: Sprawdzić i ewentualnie rozpisać dowody własności metryki
Powyższa definicja spełnia następujące własności metryki:
\begin{itemize}
    \item \textbf{Identyczność nierozróżnialnych}: \\
    Dla dowolnych multigrafów $G_1$ oraz $G_2$, $d(G_1, G_2) = 0$ wtedy i tylko wtedy, gdy $G_1$ jest izomorficzny z $G_2$. Wynika to bezpośrednio z definicji naszej metryki.
    \item \textbf{Symetria}: \\
    Dla dowolnych multigrafów $G_1$ oraz $G_2$, $d(G_1, G_2) = d(G_2, G_1)$. Dodawanie i usuwanie krawędzi lub wierzchołków jest operacją odwracalną, więc liczba operacji potrzebnych do przekształcenia $G_1$ w $G_2$ jest równa liczbie odwrotnych operacji potrzebnych do przekształcenia $G_2$ w $G_1$.
    \item \textbf{Nierówność trójkąta}: \\
    Dla dowolnych multigrafów $G_1$, $G_2$ oraz $G_3$, $d(G_1, G_3) \leq d(G_1, G_2) + d(G_2, G_3)$. Oznacza to, że najkrótsza droga między dwoma multigrafami nie może być dłuższa niż droga przechodząca przez trzeci multigraf. Jest to prawda, ponieważ każda sekwencja operacji przekształcających $G_1$ w $G_2$ oraz $G_2$ w $G_3$ może być złożona w jedną sekwencję przekształcającą $G_1$ w $G_3$.
\end{itemize}

\pagebreak
% ---------------- METRYKA W ZBIORZE WSZYSTKICH MULTIGRAFÓW ---------------- %

% ---------------- MINIMALNE ROZSZERZENIE MULTIGRAFU ---------------- %
\section{Minimalne rozszerzenie multigrafu}
\label{sec:minimalne_rozszerzenie}

\subsection{Algorytm dokładny dla problemu izomorfizmu podgrafu}
Mając dane dwa grafy G i H, chcemy znaleźć podgrafy G izomorficzne do H. Do rozwiązania tego problemu posłuży nam algorytm, który wykorzystuje procedurę Backtrackingu do sprawdzania struktury grafów. \\
Przed przejściem do algorytmu, zdefiniujmy sobie struktury przydatne nam do implementacji. Niech $n_G = |V(G)|$ - ilość wierzchołków w grafie G oraz $n_H = |V(H)|$ - ilość wierzchołków w Grafie H. \\ \\
\textbf{Opis algorytmu:}
\begin{enumerate}
    \item Inicjalizacja macierzy sąsiedztwa grafów G i H odpowiednio $S_G \in \mathbb{N}^{n_G  \times  n_G}$ i $S_H \in \mathbb{N}^{n_H  \times  n_H}$. Wartość $S[i,j]$, to ilość krawędzi pomiędzy i-tym, a j-tym wierzchołkiem dla danego grafu.
    \item Inicjalizacja kandydatów - Zdefiniujmy sobie listę $mozliwe\_dopasowania$, \\ $len(mozliwe\_dopasowania) = n_H$, gdzie pod i-tym indeksem, będziemy mieli listę możliwych dopasowań dla wierzchołka $i \in V(H)$. \\
    Algorytm Ullmana dla grafów prostych zakłada inicjalizację: \\
    $u\in V(G),  u \in mozliwe\_dopasowania[i] \iff deg_G(u) \ge deg_H(i) $ \\
    Jest ona działającą inicjalizacją dla multigrafów, jednak w celach optymalizacji algorytmu, możemy zmienić tę inicjalizację tak, aby zmniejszyć liczbę potencjalnych dopasowań, a co za tym idzie zmniejszyć liczbę gałęzi, które będzie musiał przejść algorytm. Możemy zauważyć, że w macierzach sąsiedztwa na głównej przekątnej pod indeksami $[i,i]$ znajduje się liczba pętli danego wierzchołka, zatem naszym warunkiem będzie także $S_G[i,i] \ge S_H[i,i]$. Biorąc to wszystko razem, otrzymujemy \\
    $u \in mozliwe\_dopasowania[i] \iff (deg_G(u) \ge deg_H(i)) \land  (S_G[i,i] \ge S_H[i,i]))$
    \item 
    Dla każdej krawędzi, która istnieje między już dopasowanymi wierzchołkami z H, sprawdź czy istnieje krawędź między ich dopasowaniami z G i czy ilość krawędzi między dopasowaniami jest większa lub równa niż ilość krawędzi między wierzchołkami. Można to osiągnąć przez przejrzenie wszystkich par już dopasowanych wierzchołków i krawędzi między nimi. Jeśli nie, zwróć False
    \item 
    Sprawdź, czy wszystkie wierzchołki nie zostały już dopasowane. Jeśli tak, zwróć True.
    \item 
    Dla każdego wierzchołka $v \in mozliwe\_dopasowania[i]$, jeśli $v$ $not$ $in$ $dopasowania$, przypisz $dopasowania[i] = v$ oraz wywołaj funkcję ponownie dla następnego wierzchołka $\in V(H)$ z przekazaną kopią. W przypadku wyniku True z tej funkcji, zwróć True, w przypadku False, $dopasowania[i] = null$  i przejdź do następnego kroku tej pętli.
    \item W przypadku niedopasowania po wszystkich iteracjach pętli, zwróć False.
    
\end{enumerate}

\subsubsection{Dowód poprawności}
Najpierw zbadajmy, czy algorytm dobrze inicjalizuje $mozliwe\_dopasowania$. W tym celu rozbijmy wszystkie 3 warunki. Pierwszy warunek mówi o tym, że potencjalne dopasowanie $v$ dla wierzchołka $u$, musi mieć stopień co najmniej równy stopniowi wierzchołka $u$. Gdyby tak nie było, w grafie $G$ nie istniałaby co najmniej jedna krawędź wychodząca z $v$, która istniałaby w H i wychodziłaby z $u$, zatem $v$ nie mogłoby być dopasowaniem dla $u$. 
Drugi warunek mówi o tym, że liczba pętli dla $v$ musi być co najmniej równa liczbie pętli dla $u$. Idea jest taka sama jak warunku pierwszego, gdyby warunek nie był spełniony, nie istniałaby co najmniej jedna pętla da danego wierzchołka, a co za tym idzie, nie mógłby on być dopasowaniem dla $u$.

Dalej w algorytmie, przechodzimy po kolei po wierzchołkach z H. Najpierw sprawdzamy, czy struktura się zgadza dla tych wierzchołków, do których znaleźliśmy już dopasowania. Jeśli choć 1 krawędź istniejąca w $H$ pomiędzy dwoma wierzchołkami nie będzie istnieć między ich dopasowaniami w $G$, algorytm wychodzi z tej ścieżki dopasowań i szuka innych, zatem działa poprawnie. 

Następnie sprawdzamy wszystkie z możliwych dopasowań dla danego wierzchołka, zatem sprawdzając tak wszystkie wierzchołki, mamy pewność, że przejdziemy po wszystkich możliwych permutacjach.

\subsubsection{Złożoność obliczeniowa}
Zauważmy, że inicjalizacja $mozliwe\_dopasowania$ w taki sposób, że dla każdego wierzchołka $u \in V(H)$ możliwym dopasowaniem są wszystkie $v \in V(G)$, to algorytm przejdzie po wszystkich poddrzewach, zatem w przypadku pesymistycznym do 1 wierzchołka wykona $n_G$ potencjalnych dopasowań, do drugiego $n_G - 1$, ..., a do $n_H$-tego, $(n_G - n_H + 1)$ dopasowań. Zatem mamy \[
\underbrace{(n_G)(n_G - 1)\dots(n_G - n_H + 1)}_{n_H\ \text{razy}} \le n_G^{n_H}
\]
 W każdej takiej pętli wykonujemy sprawdzenie, czy struktura grafu się zgadza, (krok 4). Zauważmy, że wykonamy tam $i^2$ porównań, gdzie i to indeks aktualnie obliczanego wierzchołka. Wiemy że $i < n_H$, zatem możemy ograniczyć tę operację: $i^2 < n_H^2$. W sumie możemy stwierdzić, że złożoność tego algorytmu wyniesie $O(n_G^{n_H}n_H^2)$



\pagebreak
\subsection{Aproksymacyjne minimalne rozszerzenie multigrafu}
\label{sec:minimalne_rozszerzenie_approx}

\noindent
Do problemu można zastosować pewną modyfikację algorytmu LeRP (Length-R Paths), opracowanego przez Freda W DePiero oraz Davida Krouta \cite{lerp_algorithm}. Algorytm opiera się na założeniu, że o podobieństwie strukturalnym dwóch wierzchołków można wnioskować na podstawie porównania liczby ścieżek (\textit{sygnatur}) o długości $r$ w ich sąsiedztwie. \\ \\ 
Modyfikacja algorytmu jako argumenty przyjmuje dwa multigrafy $G_1 = (V_1, E_1)$ i $G_2 = (V_2, E_2)$, maksymalną długość ścieżki $R$ oraz liczbę szukanych kopii grafu $k$. Im większa długość ścieżki $R$, tym algorytm jest dokładniejszy. Zwraca natomiast przekształcenie $f(g_{1i}) = g_{2k}$, gdzie $g_{1i} \in G_1$ i $g_{2k} \in G_2$, które opisuje najlepszy (w rozumieniu aproksymacji) wspólny podgraf $G_1$ oraz $G_2$. \\
Oznaczmy przez $H = (V_H, E_H)$ najlepszy wspólny podgraf $G_1$ oraz $G_2$. Do multigrafu $H$ należy dołożyć zbiór krawędzi $X$, tak aby grafy $G_1$ oraz $H' = (V_H, E_H \cup X)$ były izomorficzne. Wówczas multigraf $G_3 =(V_2, E_2\cup X)$ będzie minimalnym rozszerzeniem $G_2$, aby ten zawierał $G_1$ jako podgraf. Przez minimalne rozszerzenie, rozumie się minimalną liczbę dodanych krawędzi do grafu. \\ 
Następnie usuwamy z początkowego grafu $G_2$ wierzchołki podgrafu $H$ i powtarzamy proces dla grafów $G_1$ i $G'_2 = (V_2-V_H, E_2')$, aby znaleźć $k$ rozdzielnych kopii $G_1$ w grafie $G_2$.


\subsubsection{Opis algorytmu}
\noindent
Algorytm można podzielić na kilka etapów.

\begin{enumerate}
    \item W pierwszym kroku, należy wykonać transformację multigrafów wejściowych $G_1 = (V_1, E_1)$ oraz $G_2 = (V_2, E_2)$ na grafy atrybutowe $G'_1$ oraz $G'_2$. Transformacja przebiega w następujący sposób:
    \begin{itemize}
        \item Zbiory wierzchołków pozostają bez zmian ($V'_1 = V_1, V'_2 = V_2$).
        \item Dla każdej pary wierzchołków $(u, v)$ w $G_1$ (i analogicznie w $G_2$): Jeśli między $u$ a $v$ w $G_1$ istnieje $k$ równoległych krawędzi, to w $G'_1$ tworzona jest pojedyncza krawędź $(u, v)$ z atrybutem $k \in \mathbb{N^+}$.
    \end{itemize}
    \item W etapie drugim, dla obu grafów $G'_1$ i $G'_2$ obliczane są potęgi ich macierzy sąsiedztwa, odpowiednio $A^r$ i $B^r$ aż do maksymalnej długości $R$. Wartość $A_{ij}^r$ w macierzy $A^r$ reprezentuje liczbę ścieżek o długości dokładnie $r$ z wierzchołka $i$ do wierzchołka $j$ (na ścieżkach mogą się powtarzać wierzchołki i krawędzie). W kontekście omówionej transformacji, macierz $A$ jest macierzą, gdzie $A_{ij}$ przechowuje atrybut $k$ - liczbę równoległych krawędzi między wierzchołkami $i$ i $j$ w multigrafie $G_1$.
    \item Każdy wierzchołek $g_{1i} \in G'_1$ można porównać z każdym wierzchołkiem $g_{2k} \in G'_2$, stosując przykładowo podobieństwo cosinusowe między histogramem wartości w wierszu macierzy $A^r_i$ a histogramem wartości w wierszu macierzy $B^r_k$. Następnie tworzymy macierz z wartościami podobieństw między wierzchołkami. Wierzchołki najbardziej podobne zostają ziarnem mapowania. 
    \item Następnie iteracyjnie (przykładowo DFS), algorytm porównuje sąsiadów wierzchołków zmapowanych. Do następnego mapowania, wybiera tych sąsiadów, do których liczba ścieżek o długości co najwyżej $R$ jest największa i jest równa dla obu wierzchołków z dwóch grafów. Algorytm sprawdza również, czy wybrane wierzchołki nie zostały już zmapowane. 
    \item Zmapowane wierzchołki w grafie $G'_2$ są usuwane, tworząc nowy graf $G''_2$ i proces jest powtarzany dla grafów $G'_1$ oraz $G''_2$. W ten sposób znajdowane jest $k$ rozdzielnych kopii $G_1$ w minimalnym rozszerzeniu $G_2$. 
\end{enumerate}

\subsubsection{Złożoność obliczeniowa}
Złożoność pesymistyczna omówionej modyfikacji algorytmu LeRP jest wielomianowa i wynosi $O(N^2 \cdot(N+E) \cdot D^2 \cdot R \cdot k)$, gdzie:
\begin{itemize}
    \item $N$ to liczba wierzchołków w grafach (zakładając, że oba grafy mają rozmiar rzędu $N$).
    \item $E$ to liczba krawędzi w grafach (zakładając, że oba grafy mają liczbę krawędzi rzędu $N$).
    \item $D$ to średni stopień wierzchołków w grafach.
    \item $R$ to maksymalna długość ścieżki brana pod uwagę. 
    \item $k$ to liczba szukanych kopii w minimalnym rozszerzeniu
\end{itemize}
Uzasadnienie złożoności: porównanie par wierzchołków wymaga $N^2$ porównań. Każdy wierzchołek ma średnio $D$ sąsiadów, więc porównywanie sąsiadów daje $D^2$ dodatkowych operacji. Analiza różnych długości ścieżek to czynnik $R$. Podczas budowania dopasowania, algorytm iteracyjny ma złożoność $(N+E)$. Czynnik $k$ odpowiada za znalezienie $k$ kopii mniejszego multigrafu $G_1$.\\ \\
W kontekście tego algorytmu aproksymacyjnego nie rozważano formalnego dowodu poprawności. Dostarczono empiryczną gwarancję jakości - algorytm testowano na dużych zbiorach danych, wykazując, że algorytm konsekwentnie zwraca wyniki bliskie optimum w praktyce.


% ---------------- MINIMALNE ROZSZERZENIE MULTIGRAFU ---------------- %




% \section{Testy}
% \section{Podsumowanie}
\section{Bibliografia}

\bibliographystyle{plain}
\bibliography{refs}

\end{document}